%% LaTeX2e class for seminar theses
%% sections/content.tex
%% 
%% Karlsruhe Institute of Technology
%% Institute for Program Structures and Data Organization
%% Chair for Software Design and Quality (SDQ)
%%
%% Dr.-Ing. Erik Burger
%% burger@kit.edu
%%
%% Version 1.0, 2018-04-16

\section{Einf"uhrung}
Jede Programmiersprache wurde mit einem bestimmten Motto entwickelt. Dadurch zeichnet sich auch jede Programmiersprach durch eigene St"arken aus. So ist zum Beispiel C eine hervorragende Sprache, wenn es darum geht, hardwarenah und effizient zu programmieren. Verliert aber schnell an Macht, wenn es darum geht umfangreiche Business Software. Java hingegen eignet sich hervorragend f"ur solche Software, braucht allerdings eine JVM, was Java weitesgehendst untauglich macht, wenn Speicher und Rechenleistung knapp sind. Beispiele finden, wo die Kombination Sinn macht.

Aus diesem Grund kann es w"unschenswert sein, mehrere Sprachen im selben Projekt zu verwenden. Zum einen ist dies m"oglich in der Form C und Assembler, wo durch spezielle Instruktionen Assemblercode direkt in C-Code eingebettet werden kann. Allerdings gibt es f"ur andere Sprachen keine vorgesehenen Schnittstellen. Dann wird es erforderlich, einen Adapter f"ur die jeweilige Sprache zu benutzen. Diese erm"oglicht es, Methoden oder Befehle einer Sprache aus dem Code aus einer anderen Sprache heraus aufzurufen. In meinem Praktikum habe ich einen solchen Adapter f"ur das Ausf"hren von Prologanfragen in Java implementiert.

Trust 4.0 kurz einf"uhren.