%% LaTeX2e class for seminar theses
%% sections/content.tex
%% 
%% Karlsruhe Institute of Technology
%% Institute for Program Structures and Data Organization
%% Chair for Software Design and Quality (SDQ)
%%
%% Dr.-Ing. Erik Burger
%% burger@kit.edu
%%
%% Version 1.0, 2018-04-16

\section{Introduction}
\label{ch:Introduction}

%% -------------------
%% | Example content |
%% -------------------

This is the SDQ seminar template.
For more information on the formatting of theses at SDQ, please refer to
\url{https://sdqweb.ipd.kit.edu/wiki/Ausarbeitungshinweise} or to your advisor.

\subsection{Example: Citation}
\label{sec:Introduction:Citation}
A citation: \cite{becker2008a} For referencing, see \autoref{sec:Introduction:Figures}

\subsection{Example: Figures}
\label{sec:Introduction:Figures}
A reference: The SDQ logo is displayed in \autoref{fig:sdqlogo}. 
(Use \code{\textbackslash autoref\{\}} for easy referencing.) 

\subsection{Example: Tables}
\label{sec:Introduction:Tables}

\subsection{Example: Todo-Note}
Meaningless text.

\subsection{Example: Formula}
One of the nice things about the Linux Libertine font is that it comes with
a math mode package.
\begin{displaymath}
f(x)=\Omega(g(x))\ (x\rightarrow\infty)\;\Leftrightarrow\;
\limsup_{x \to \infty} \left|\frac{f(x)}{g(x)}\right|> 0
\end{displaymath}

%% --------------------
%% | /Example content |
%% --------------------