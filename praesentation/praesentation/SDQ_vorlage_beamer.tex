\documentclass[18pt]{beamer}

%% SLIDE FORMAT

% use 'beamerthemekit' for standard 4:3 ratio
% for widescreen slides (16:9), use 'beamerthemekitwide'

\usepackage{templates/beamerthemekit}
%\usepackage{templates/beamerthemekitwide}
\usepackage{amsmath}
\usepackage{amssymb}
\usepackage[ngerman]{babel}
\usepackage{graphicx}
\usepackage{color}
\usepackage[utf8]{inputenc}
\usepackage[english]{babel}
\usepackage{minted}

%% TITLE PICTURE

% if a custom picture is to be used on the title page, copy it into the 'logos'
% directory, in the line below, replace 'mypicture' with the 
% filename (without extension) and uncomment the following line
% (picture proportions: 63 : 20 for standard, 169 : 40 for wide
% *.eps format if you use latex+dvips+ps2pdf, 
% *.jpg/*.png/*.pdf if you use pdflatex)

%\titleimage{board}

%% TITLE LOGO

% for a custom logo on the front page, copy your file into the 'logos'
% directory, insert the filename in the line below and uncomment it

\titlelogo{blank}

% (*.eps format if you use latex+dvips+ps2pdf,
% *.jpg/*.png/*.pdf if you use pdflatex)

%% TikZ INTEGRATION

% use these packages for PCM symbols and UML classes
% \usepackage{templates/tikzkit}
% \usepackage{templates/tikzuml}

% the presentation starts here

\title[Prologadapter für Java]{Prologadapter für Java}
\subtitle{Betreut von Stephan Seifermann}
\author{Johannes Werner}

\institute{Fakultaet für Informatik}

% Bibliography

\usepackage[citestyle=authoryear,bibstyle=numeric,hyperref,backend=biber]{biblatex}
\addbibresource{templates/example.bib}
\bibhang1em
{\renewcommand{\arraystretch}{1.3}
\begin{document}

% change the following line to "ngerman" for German style date and logos
\selectlanguage{ngerman}

%title page
\begin{frame}
\titlepage
\end{frame}

%table of contents
\begin{frame}{Gliederung}
\tableofcontents
\end{frame}

\section{Motivation}
\begin{frame}{Motivation}
\begin{itemize}
\item Sprachen haben unterschiedliche St"arken
\begin{itemize}
\item z.B. Haskell vs. C
\end{itemize}
\item Wie kann man diese St"arken kombinieren?
\item Wie kann man mehrere Sprachen im selben Programm nutzen?
\item \textbf{L"osung:} Adapter einsetzen
\end{itemize}
\begin{figure}[h]
\centering
\includegraphics[width=0.95\textwidth]{adapter.pdf}
\end{figure}
%TODO Realworld Beispiel bringen in der Praesentation
\end{frame}
\begin{frame}{Projektbezug}
\begin{itemize}
\item Softwaresystem soll Datenschutz garantieren
\item Architekturmodell wird um Datenflussannotationen erg"anzt
\item Durch Analyse k"onnen Risiken gefunden werden
\vspace{1cm}
\item Architekturmodell
\begin{itemize}
\item Eclispe-Plugin in \textbf{Java}
\end{itemize}
\item Datenflussanalyse
\begin{itemize}
\item Datenflussmodell wird zu \textbf{Prolog}-Statements transformiert und ausgewertet
\end{itemize}
\item $\Rightarrow$ \textbf{Java-zu-Prolog Adapter}
\end{itemize}
\end{frame}

\section{Anforderungen}
\begin{frame}{Anforderungen}
\begin{itemize}
\item Anbindung zu mehreren Prolog-Interpreter
\begin{itemize}
\item Mndst. 1 in Java implementierter Interpreter
%TODO Grund in Praesi erlaeutern
\end{itemize}
\vspace{0.3cm}
\item Interpreter muss EPL kompatible Lizenz haben
\vspace{0.3cm}
\item Java-Objekte als Antwort vom Interpreter
\begin{itemize}
\item Anfrage: String('2+2')
\item Antwort: Integer(4)
\end{itemize}
\end{itemize}
\end{frame}

\section{Interpreterwahl}
\begin{frame}{Interpreterwahl}
\begin{itemize}
\item \textbf{Kriterien} (* Pflicht)
\begin{itemize}
\item Java-Anbindung*
\item Passende Lizenz*
\item Paket im Maven-Repository
\item In Java implementiert
\item Zeitpunkt des letzten Releases
\end{itemize}
\vspace{0.5cm}
\item \textbf{Herausforderung}
\begin{itemize}
\item Prolog Community ist klein, weshalb es nicht viele aktive Projekte gibt
\end{itemize}
\end{itemize}
\end{frame}
\begin{frame}{Interpreterwahl}
\begin{table}[]
\resizebox{\textwidth}{!}{%
\begin{tabular}{r|ccc}
\multicolumn{1}{l|}{} & \textbf{SWI-Prolog} & \textbf{TuProlog} & \textbf{Projog} \\ \hline
Lizenz                & BSD          & LGPL              & Apache 2.0      \\ \hline
Letzter Release       & Juli'19          & Okt'18        & Dez'18      \\ \hline
Maven-Paket           & \checkmark          & \checkmark        & \checkmark      \\ \hline
In Java implementiert & $\times$            & \checkmark        & \checkmark     
\end{tabular}%
}
\end{table}
\vspace{0.5cm}
\begin{itemize}
\item SWI-Prolog ist der weitestverbreitetste und am aktivsten weiterentwickelte Interpreter
\end{itemize}
\end{frame}

\section{Java-Schnittstelle}
\begin{frame}[fragile]{Java-Schnittstelle}
Beispiel Prologdatenbank:
\begin{minted}{Prolog}
child_of(Hans, Peter).
child_of(Brigitte, Peter).
\end{minted}
\vspace{1cm}
\textbf{Ziel:} Prolog-Statement in Java ausf"uhren
\begin{minted}{Prolog}
    child_of(Norbert, Peter).
\end{minted}
\end{frame}
\begin{frame}[fragile]{Java-Schnittstelle}
\begin{minted}{Java}
class Human {
	private String name;
	// Getter and Setter are present
}

public boolean isChildOf(Human parent, Human childToCheck) {
	// How to query the prolog database?
	// The query to execute:
	//     child_of(nameOfParent, nameOfChildToCheck)
}
\end{minted}
\end{frame}
\begin{frame}[fragile]{Java-Schnittstelle SWI-Prolog}
\begin{minted}{Java}
public boolean isChildOf(Human parent, Human childToCheck) {
    Atom parentName = convert(parent);
    Atom childName = convert(childToCheck);
    // goal = child_of(parentName, childName)
    Term goal = 
        Util.textParamsToTerm(
            "child_of(?parent, ?child).", 
            parentName, childName);
    Query query = new Query(goal);
    return query.hasMoreSolutions();
}
\end{minted}
\end{frame}
\begin{frame}[fragile]{Java-Schnittstelle Projog}
\begin{minted}{Java}
private Projog engine;

public boolean isChildOf(Human parent, Human childToCheck) {
    String pName = parent.getName();
    String cName = childToCheck.getName();
    QueryResult solution =
        engine.solve("child_of("+pName+","+cName+").");
    return solution.next();
}
\end{minted}
\end{frame}

\section{Prolog4J}
\begin{frame}{Prolog4J}
\begin{itemize}
\item Java-zu-Prolog Adapter
\item MIT Lizenz (EPL kompatibel)
\item Unterst"utzte Interpreter
\begin{itemize}
\item SWI-Prolog
\item TuProlog
\item JTrolog
\item JLog
\end{itemize}
\item Letzter Commit: \textbf{Januar 2011}
\end{itemize}
\end{frame}
\subsection{Architektur}
\begin{frame}[fragile]{Architektur}
\begin{minted}{Java}
private Prover prover;

public boolean isChildOf(Human parent, Human childToCheck) {
    Query query = p.query("child_of(?parent, ?child).");
    Solution<?> solution = 
        query.solve(parent, childToCheck);
    return solution.isSuccess();
}
\end{minted}
\end{frame}
\begin{frame}{Architektur}
\begin{figure}[h]
\centering
\includegraphics[width=1\textwidth]{prolog4j.pdf}
\end{figure}
\end{frame}



\subsection{Anpassungen}
\begin{frame}{Anpassungen}
%TODO in Praesi auf jeden Punkt genauer eingehen
\begin{itemize}
\item Entfernen ungewollter Interpreter
\item Aktualisieren der POM-Dateien
\begin{itemize}
\item Maven als Build-System
\end{itemize}
\item Aktualisieren des Codes
\item Projog Interpreter hinzuf"ugen
\item API um n"otige Funktionalit"at erweitern
\item Interpreter-Discovery mit OSGI
\end{itemize}
\end{frame}
\begin{frame}[fragile]{SWI Prolog Anpassung}
\begin{itemize}
\item Darstellung einer Liste in Standard Prolog
\begin{minted}{Prolog}
'.'([],[])
\end{minted}
\item SWI-Prolog Darstellung
\begin{minted}{Prolog}
'[|]'([],[])
\end{minted}
\vspace{0.5cm}
\item Begr"undung
\begin{minted}{Text}
As of version 7, SWI-Prolog lists can be 
distinguished unambiguously at runtime
from ./2 terms and the atom '[]'
\end{minted}
\end{itemize}
\end{frame}
\begin{frame}[fragile]{Bug in Projog}
\begin{itemize}
\item Neu hinzugef"ugte Regeln f"ur einen Funktor "uberschreiben vorherige Regeln
\begin{itemize}
\item Regeln werden in einer Hashmap gespeichert und "uberschreiben beim Einf"ugen den vorherigen Wert des Funktors
\end{itemize}
\vspace{0.5cm}
\item Folgende Regeln sollen der Datenbank hinzugef"ugt werden
\begin{minted}{Prolog}
human(socrates), human(euklid), human(plato)
\end{minted}
\end{itemize}
\end{frame}
\begin{frame}[fragile]{Bug in Projog}
\begin{itemize}
\item Folgender Code funktioniert
\begin{minted}{Prolog}
addTheory(human(socrates), human(euklid)), human(plato)).
\end{minted}
\vspace{0.5cm}
\item Resultierende Datenbank
\begin{minted}{Prolog}
human(socrates)
human(euklid)
human(plato)
\end{minted}
\end{itemize}
\end{frame}
\begin{frame}[fragile]{Bug in Projog}
\begin{itemize}
\item Folgender Code funktioniert \textbf{nicht}
\begin{minted}{Prolog}
addTheory(human(socrates)).
addTheory(human(euklid)).
addTheory(human(plato)).
\end{minted}
\vspace{0.5cm}
\item Resultierende Datenbank
\begin{minted}{Prolog}
human(plato)
\end{minted}
\end{itemize}
\end{frame}
\begin{frame}{Bug in Projog - L"osung}
\begin{itemize}
\item Eine Liste mit allen Regeln per Funktor wird parallel gehalten
\vspace{0.5cm}
\item Einf"ugen einer Regel:
\begin{enumerate}
\item Einf"ugen der Regel in die eigene Datenbank
\item Einf"ugen der gesamten Datenbank in die Projog-Datenbank
\end{enumerate}
\end{itemize}
\end{frame}
\begin{frame}{Interpreter-Discovery mit OSGI}
\begin{itemize}
\item \textbf{O}pen \textbf{S}ervices \textbf{G}ateway \textbf{i}nitiative
\begin{itemize}
\item Einbinden modularer Services zur Laufzeit
\end{itemize}
\vspace{0.1cm}
\item Erm"oglicht Laufzeiterkennung der verf"ugbaren Interpreter
\begin{itemize}
\item Kein festes Einprogrammieren n"otig
\end{itemize}
\end{itemize}
\end{frame}
\begin{frame}{Interpreter-Discovery mit OSGI}
\begin{figure}[h]
\centering
\includegraphics[width=1\textwidth]{osgi.pdf}
\end{figure}
\end{frame}


\begin{frame}{Zusammenfassung}
\begin{itemize}
\item Ausf"uhren von Prolog-Befehlen in Java
\item Adapter soll an mehrere Interpreter anbinden
\begin{itemize}
\item SWI-Prolog
\item TuProlog
\item Projog
\end{itemize}
\item Java-Schnittstellen der Interpreter sind unterschiedlich
\begin{itemize}
\item Gutes Design des Adapters erforderlich
\end{itemize}
\item Existierende Bibliothek Prolog4J bietet was wir brauchen
\begin{itemize}
\item Muss modernisiert werden
\item Hinzuf"ugen von OSGI Discovery
\end{itemize}
\end{itemize}
\end{frame}

\end{document}
